% Options for packages loaded elsewhere
\PassOptionsToPackage{unicode}{hyperref}
\PassOptionsToPackage{hyphens}{url}
%
\documentclass[
  11pt]{report}
\usepackage{amsmath,amssymb}
\usepackage{lmodern}
\usepackage{iftex}
\ifPDFTeX
  \usepackage[T1]{fontenc}
  \usepackage[utf8]{inputenc}
  \usepackage{textcomp} % provide euro and other symbols
\else % if luatex or xetex
  \usepackage{unicode-math}
  \defaultfontfeatures{Scale=MatchLowercase}
  \defaultfontfeatures[\rmfamily]{Ligatures=TeX,Scale=1}
\fi
% Use upquote if available, for straight quotes in verbatim environments
\IfFileExists{upquote.sty}{\usepackage{upquote}}{}
\IfFileExists{microtype.sty}{% use microtype if available
  \usepackage[]{microtype}
  \UseMicrotypeSet[protrusion]{basicmath} % disable protrusion for tt fonts
}{}
\makeatletter
\@ifundefined{KOMAClassName}{% if non-KOMA class
  \IfFileExists{parskip.sty}{%
    \usepackage{parskip}
  }{% else
    \setlength{\parindent}{0pt}
    \setlength{\parskip}{6pt plus 2pt minus 1pt}}
}{% if KOMA class
  \KOMAoptions{parskip=half}}
\makeatother
\usepackage{xcolor}
\IfFileExists{xurl.sty}{\usepackage{xurl}}{} % add URL line breaks if available
\IfFileExists{bookmark.sty}{\usepackage{bookmark}}{\usepackage{hyperref}}
\hypersetup{
  pdftitle={Introduction to Probability (Joe Blitzstein)},
  pdfauthor={Fernando Náufel},
  pdflang={pt-br},
  hidelinks,
  pdfcreator={LaTeX via pandoc}}
\urlstyle{same} % disable monospaced font for URLs
\usepackage[margin=1in]{geometry}
\usepackage{color}
\usepackage{fancyvrb}
\newcommand{\VerbBar}{|}
\newcommand{\VERB}{\Verb[commandchars=\\\{\}]}
\DefineVerbatimEnvironment{Highlighting}{Verbatim}{commandchars=\\\{\}}
% Add ',fontsize=\small' for more characters per line
\usepackage{framed}
\definecolor{shadecolor}{RGB}{248,248,248}
\newenvironment{Shaded}{\begin{snugshade}}{\end{snugshade}}
\newcommand{\AlertTok}[1]{\textcolor[rgb]{0.94,0.16,0.16}{#1}}
\newcommand{\AnnotationTok}[1]{\textcolor[rgb]{0.56,0.35,0.01}{\textbf{\textit{#1}}}}
\newcommand{\AttributeTok}[1]{\textcolor[rgb]{0.77,0.63,0.00}{#1}}
\newcommand{\BaseNTok}[1]{\textcolor[rgb]{0.00,0.00,0.81}{#1}}
\newcommand{\BuiltInTok}[1]{#1}
\newcommand{\CharTok}[1]{\textcolor[rgb]{0.31,0.60,0.02}{#1}}
\newcommand{\CommentTok}[1]{\textcolor[rgb]{0.56,0.35,0.01}{\textit{#1}}}
\newcommand{\CommentVarTok}[1]{\textcolor[rgb]{0.56,0.35,0.01}{\textbf{\textit{#1}}}}
\newcommand{\ConstantTok}[1]{\textcolor[rgb]{0.00,0.00,0.00}{#1}}
\newcommand{\ControlFlowTok}[1]{\textcolor[rgb]{0.13,0.29,0.53}{\textbf{#1}}}
\newcommand{\DataTypeTok}[1]{\textcolor[rgb]{0.13,0.29,0.53}{#1}}
\newcommand{\DecValTok}[1]{\textcolor[rgb]{0.00,0.00,0.81}{#1}}
\newcommand{\DocumentationTok}[1]{\textcolor[rgb]{0.56,0.35,0.01}{\textbf{\textit{#1}}}}
\newcommand{\ErrorTok}[1]{\textcolor[rgb]{0.64,0.00,0.00}{\textbf{#1}}}
\newcommand{\ExtensionTok}[1]{#1}
\newcommand{\FloatTok}[1]{\textcolor[rgb]{0.00,0.00,0.81}{#1}}
\newcommand{\FunctionTok}[1]{\textcolor[rgb]{0.00,0.00,0.00}{#1}}
\newcommand{\ImportTok}[1]{#1}
\newcommand{\InformationTok}[1]{\textcolor[rgb]{0.56,0.35,0.01}{\textbf{\textit{#1}}}}
\newcommand{\KeywordTok}[1]{\textcolor[rgb]{0.13,0.29,0.53}{\textbf{#1}}}
\newcommand{\NormalTok}[1]{#1}
\newcommand{\OperatorTok}[1]{\textcolor[rgb]{0.81,0.36,0.00}{\textbf{#1}}}
\newcommand{\OtherTok}[1]{\textcolor[rgb]{0.56,0.35,0.01}{#1}}
\newcommand{\PreprocessorTok}[1]{\textcolor[rgb]{0.56,0.35,0.01}{\textit{#1}}}
\newcommand{\RegionMarkerTok}[1]{#1}
\newcommand{\SpecialCharTok}[1]{\textcolor[rgb]{0.00,0.00,0.00}{#1}}
\newcommand{\SpecialStringTok}[1]{\textcolor[rgb]{0.31,0.60,0.02}{#1}}
\newcommand{\StringTok}[1]{\textcolor[rgb]{0.31,0.60,0.02}{#1}}
\newcommand{\VariableTok}[1]{\textcolor[rgb]{0.00,0.00,0.00}{#1}}
\newcommand{\VerbatimStringTok}[1]{\textcolor[rgb]{0.31,0.60,0.02}{#1}}
\newcommand{\WarningTok}[1]{\textcolor[rgb]{0.56,0.35,0.01}{\textbf{\textit{#1}}}}
\usepackage{longtable,booktabs,array}
\usepackage{calc} % for calculating minipage widths
% Correct order of tables after \paragraph or \subparagraph
\usepackage{etoolbox}
\makeatletter
\patchcmd\longtable{\par}{\if@noskipsec\mbox{}\fi\par}{}{}
\makeatother
% Allow footnotes in longtable head/foot
\IfFileExists{footnotehyper.sty}{\usepackage{footnotehyper}}{\usepackage{footnote}}
\makesavenoteenv{longtable}
\setlength{\emergencystretch}{3em} % prevent overfull lines
\providecommand{\tightlist}{%
  \setlength{\itemsep}{0pt}\setlength{\parskip}{0pt}}
\setcounter{secnumdepth}{5}
\newlength{\cslhangindent}
\setlength{\cslhangindent}{1.5em}
\newlength{\csllabelwidth}
\setlength{\csllabelwidth}{3em}
\newlength{\cslentryspacingunit} % times entry-spacing
\setlength{\cslentryspacingunit}{\parskip}
\newenvironment{CSLReferences}[2] % #1 hanging-ident, #2 entry spacing
 {% don't indent paragraphs
  \setlength{\parindent}{0pt}
  % turn on hanging indent if param 1 is 1
  \ifodd #1
  \let\oldpar\par
  \def\par{\hangindent=\cslhangindent\oldpar}
  \fi
  % set entry spacing
  \setlength{\parskip}{#2\cslentryspacingunit}
 }%
 {}
\usepackage{calc}
\newcommand{\CSLBlock}[1]{#1\hfill\break}
\newcommand{\CSLLeftMargin}[1]{\parbox[t]{\csllabelwidth}{#1}}
\newcommand{\CSLRightInline}[1]{\parbox[t]{\linewidth - \csllabelwidth}{#1}\break}
\newcommand{\CSLIndent}[1]{\hspace{\cslhangindent}#1}
\ifLuaTeX
\usepackage[bidi=basic]{babel}
\else
\usepackage[bidi=default]{babel}
\fi
\babelprovide[main,import]{brazilian}
% get rid of language-specific shorthands (see #6817):
\let\LanguageShortHands\languageshorthands
\def\languageshorthands#1{}

% A command to save the path to the resources of bd.format (fnaufel)
\newcommand{\dir}{/ssd/R/x86_64-pc-linux-gnu-library/4.1/fnaufelRmd/rmarkdown/resources}



\hypersetup{
  colorlinks,
  breaklinks,
  linkcolor=magenta,
  urlcolor=blue
}

% Lexend font
\usepackage{lexend}


% Para bibliografia em português
\usepackage{babelbib}

% Para títulos de capítulos e seções:
\usepackage[nobottomtitles*]{titlesec}

%%%%%%%%%%%%%%%
%
% Titulos de capítulos e seções

\titleformat{\chapter}[display]%
{\bfseries\Large}%
{\filleft\MakeUppercase{\chaptertitlename} \Huge\thechapter}%
{4ex}%
{\titlerule%
  \vspace{2ex}%
  \filright}%
[\vspace{2ex}%
\titlerule%
\vspace{10ex}]

\titleformat{\section}[block]%
{\bfseries\Large}%
{\thesection}{.5em}{\titlerule\\[.8ex]\bfseries}

\titleformat{\subsection}[block]%
{\bfseries}%
{\thesubsection}{.5em}{\titlerule\\[.8ex]\bfseries}%
[\vspace{1ex}]

%%%%%%%%%%%%%%%
%
% Caixas

\usepackage{tcolorbox}

\tcbset{
  rounded corners,
  boxrule=0.3mm,
  colback=black!.5!white,
  parbox=false
}

\newtcolorbox{rmdbox}{
  colframe=black!40!white,
}


\newtcolorbox{mycaution}{
  colframe=red!75!black,
  sidebyside,
  lower separated=false,
  lefthand width=1cm,
  sidebyside gap=4mm
}

\newenvironment{rmdcaution}
{
  \begin{mycaution}
    \includegraphics[width=.8cm]{\dir/images/caution.png}
    \tcblower
  }
  {
  \end{mycaution}
}

\newtcolorbox{myimportant}{
  colframe=green!75!black,
  sidebyside,
  lower separated=false,
  lefthand width=1cm,
  sidebyside gap=4mm
}

\newenvironment{rmdimportant}
{
  \begin{myimportant}
    \includegraphics[width=.8cm]{\dir/images/important.png}
    \tcblower
  }
  {
  \end{myimportant}
}

\newtcolorbox{mywarning}{
  colframe=yellow!80!black,
  sidebyside,
  lower separated=false,
  lefthand width=1cm,
  sidebyside gap=4mm
}

\newenvironment{rmdwarning}
{
  \begin{mywarning}
    \includegraphics[width=.8cm]{\dir/images/warning.png}
    \tcblower
  }
  {
  \end{mywarning}
}

\newtcolorbox{mynote}{
  colframe=yellow!70!black,
  sidebyside,
  lower separated=false,
  lefthand width=1cm,
  sidebyside gap=4mm
}

\newenvironment{rmdnote}
{
  \begin{mynote}
    \includegraphics[width=.8cm]{\dir/images/note.png}
    \tcblower
  }
  {
  \end{mynote}
}

\newtcolorbox{mytip}{
  colframe=blue!50!white,
  sidebyside,
  lower separated=false,
  lefthand width=1cm,
  sidebyside gap=4mm
}

\newenvironment{rmdtip}
{
  \begin{mytip}
    \includegraphics[width=.8cm]{\dir/images/tip.png}
    \tcblower
  }
  {
  \end{mytip}
}

% For highlighting using \hl{}
\usepackage{soul}


\makeatletter
\@ifundefined{Shaded}{}{
  % Code chunks and output
  \usepackage[framemethod=pgf]{mdframed}
  \renewenvironment{Shaded}{
    \begin{mdframed}[%
      roundcorner=2pt,%
      innerleftmargin=5pt,%
      innerrightmargin=5pt,%
      topline=true,%
      leftline=true,%
      rightline=true,%
      bottomline=true,%
      linewidth=0.5pt,%
      linecolor=black!20,%
      backgroundcolor=black!2,%
      skipabove=2ex,%
      skipbelow=2.5ex%
    ]%
  }
  {
    \end{mdframed}
  }
}
\makeatother

% End of preamble for bookdowntemplate01

%%%%%%%%%%%%%%%%%%%%%%%%%%%%%%%%%%%%%%%%%%%%%%%%%%%%%%

\usepackage{booktabs}
\usepackage{longtable}
\usepackage{array}
\usepackage{multirow}
\usepackage{wrapfig}
\usepackage{float}
\usepackage{colortbl}
\usepackage{pdflscape}
\usepackage{tabu}
\usepackage{threeparttable}
\usepackage{threeparttablex}
\usepackage[normalem]{ulem}
\usepackage{makecell}
\usepackage{xcolor}
\ifLuaTeX
  \usepackage{selnolig}  % disable illegal ligatures
\fi

\title{Introduction to Probability (Joe Blitzstein)}
\author{Fernando Náufel}
\date{(versão de 04/02/2022)}

\begin{document}
\maketitle

{
\setcounter{tocdepth}{1}
\tableofcontents
}
\hypertarget{apresentauxe7uxe3o}{%
\chapter*{Apresentação}\label{apresentauxe7uxe3o}}
\addcontentsline{toc}{chapter}{Apresentação}

\begin{itemize}
\item
  Página do livro: \url{https://projects.iq.harvard.edu/stat110/home}
\item
  Strategic practice and homework: \url{https://projects.iq.harvard.edu/stat110/strategic-practice-problems}
\item
  Handouts: \url{https://projects.iq.harvard.edu/stat110/handouts} --- includes solutions to exercises marked with (s) in the book.
\item
  Playlist: \url{https://www.youtube.com/playlist?list=PL2SOU6wwxB0uwwH80KTQ6ht66KWxbzTIo}
\end{itemize}

\hypertarget{probabilidade-e-contagem}{%
\chapter*{01: Probabilidade e contagem}\label{probabilidade-e-contagem}}
\addcontentsline{toc}{chapter}{01: Probabilidade e contagem}

\hypertarget{vuxeddeo}{%
\section*{Vídeo}\label{vuxeddeo}}
\addcontentsline{toc}{section}{Vídeo}

\begin{center} \url{https://youtu.be/KbB0FjPg0mw} \end{center}

\hypertarget{pascal-e-fermat}{%
\section*{Pascal e Fermat}\label{pascal-e-fermat}}
\addcontentsline{toc}{section}{Pascal e Fermat}

\begin{itemize}
\item
  Ver artigo DEVLIN (\protect\hyperlink{ref-devlin-2010-pascal-fermat}{2010}).
\item
  Ver \href{https://gallica.bnf.fr/ark:/12148/bpt6k69975r.image.r=Blaise+Pascal.f233.langFR}{originais em francês de toda a correspondência de Pascal}.
\end{itemize}

\hypertarget{r}{%
\section*{R}\label{r}}
\addcontentsline{toc}{section}{R}

\hypertarget{fatoriais-e-combinauxe7uxf5es}{%
\subsection*{Fatoriais e combinações}\label{fatoriais-e-combinauxe7uxf5es}}
\addcontentsline{toc}{subsection}{Fatoriais e combinações}

\begin{itemize}
\item
  Qual o maior valor de $n$ para o qual o R calcula \texttt{factorial(n)}?

  No meu computador, $n = 170$:

\begin{Shaded}
\begin{Highlighting}[]
\FunctionTok{factorial}\NormalTok{(}\DecValTok{170}\SpecialCharTok{:}\DecValTok{171}\NormalTok{)}
\DocumentationTok{\#\# [1] 7,257416e+306           Inf}
\end{Highlighting}
\end{Shaded}
\item
  Para valores maiores, podemos usar \texttt{lfactorial(n)} para calcular $\ln n!$:

\begin{Shaded}
\begin{Highlighting}[]
\FunctionTok{lfactorial}\NormalTok{(}\DecValTok{170}\SpecialCharTok{:}\DecValTok{171}\NormalTok{)}
\DocumentationTok{\#\# [1] 706,5731 711,7147}
\end{Highlighting}
\end{Shaded}
\item
  Da mesma forma, \texttt{lchoose(n,\ k)} calcula $\ln \binom nk$.
\end{itemize}

\hypertarget{tabulando-dados-tabulate-times-table}{%
\subsection*{\texorpdfstring{Tabulando dados: \texttt{tabulate} $\times$ \texttt{table}}{Tabulando dados: tabulate  table}}\label{tabulando-dados-tabulate-times-table}}
\addcontentsline{toc}{subsection}{Tabulando dados: \texttt{tabulate} $\times$ \texttt{table}}

\begin{Shaded}
\begin{Highlighting}[]
\NormalTok{b }\OtherTok{\textless{}{-}} \FunctionTok{sample}\NormalTok{(}\DecValTok{1}\SpecialCharTok{:}\DecValTok{365}\NormalTok{,}\DecValTok{23}\NormalTok{,}\AttributeTok{replace=}\ConstantTok{TRUE}\NormalTok{)}
\FunctionTok{tabulate}\NormalTok{(b)}
\DocumentationTok{\#\#   [1] 0 0 0 0 0 0 0 0 0 0 0 0 0 0 0 0 0 0 0 0 0 0 0 0 0 1 1 0 0 0 0 0 0}
\DocumentationTok{\#\#  [34] 0 0 0 0 0 0 0 0 0 0 0 1 0 0 0 0 0 0 0 0 0 0 0 0 0 0 0 0 1 0 0 0 0}
\DocumentationTok{\#\#  [67] 0 1 0 0 0 0 0 0 0 0 0 0 0 0 0 0 0 0 0 0 0 0 0 0 0 0 1 0 0 0 0 0 0}
\DocumentationTok{\#\# [100] 0 0 0 0 0 0 0 1 0 0 0 0 0 0 0 0 0 0 0 0 0 0 0 0 0 0 0 0 0 0 0 0 0}
\DocumentationTok{\#\# [133] 0 0 0 0 0 0 0 1 0 0 0 0 0 0 0 0 0 0 0 0 0 0 0 0 1 0 0 1 1 0 0 0 0}
\DocumentationTok{\#\# [166] 0 0 0 0 0 0 1 0 0 0 0 0 0 0 0 0 0 0 0 0 0 0 0 0 0 0 0 0 0 0 0 0 0}
\DocumentationTok{\#\# [199] 0 0 0 0 0 0 0 1 0 1 0 0 0 0 0 1 0 0 0 0 0 0 0 0 0 0 0 0 0 0 0 0 0}
\DocumentationTok{\#\# [232] 0 0 1 0 0 0 0 0 0 0 0 0 0 0 0 0 0 0 0 0 0 0 0 0 0 0 0 0 0 0 0 0 0}
\DocumentationTok{\#\# [265] 1 0 0 0 0 0 1 0 0 0 0 0 0 0 0 0 0 0 0 1 0 0 0 0 0 0 0 0 0 0 0 0 0}
\DocumentationTok{\#\# [298] 1 0 0 0 0 0 0 0 0 0 0 0 1 0 0 0 0 0 0 0 0 1 0 0 0 0 0 0 0 0 0 0 0}
\DocumentationTok{\#\# [331] 0 0 0 0 0 0 0 0 0 0 0 0 0 0 0 0 0 1}
\end{Highlighting}
\end{Shaded}

\begin{Shaded}
\begin{Highlighting}[]
\FunctionTok{table}\NormalTok{(b)}
\DocumentationTok{\#\# b}
\DocumentationTok{\#\#  26  27  45  62  68  93 107 140 157 160 161 172 206 208 214 234 265 }
\DocumentationTok{\#\#   1   1   1   1   1   1   1   1   1   1   1   1   1   1   1   1   1 }
\DocumentationTok{\#\# 271 284 298 310 319 348 }
\DocumentationTok{\#\#   1   1   1   1   1   1}
\end{Highlighting}
\end{Shaded}

\hypertarget{funuxe7uxf5es-para-o-problema-do-aniversuxe1rio}{%
\subsection*{Funções para o problema do aniversário}\label{funuxe7uxf5es-para-o-problema-do-aniversuxe1rio}}
\addcontentsline{toc}{subsection}{Funções para o problema do aniversário}

\begin{Shaded}
\begin{Highlighting}[]
\FunctionTok{pbirthday}\NormalTok{(}\DecValTok{23}\NormalTok{)}
\DocumentationTok{\#\# [1] 0,5072972}
\end{Highlighting}
\end{Shaded}

\begin{Shaded}
\begin{Highlighting}[]
\FunctionTok{qbirthday}\NormalTok{(.}\DecValTok{5}\NormalTok{)}
\DocumentationTok{\#\# [1] 23}
\end{Highlighting}
\end{Shaded}

\begin{Shaded}
\begin{Highlighting}[]
\FunctionTok{qbirthday}\NormalTok{(}\DecValTok{1}\NormalTok{)}
\DocumentationTok{\#\# [1] 366}
\end{Highlighting}
\end{Shaded}

Para no mínimo $3$ no mesmo dia:

\begin{Shaded}
\begin{Highlighting}[]
\FunctionTok{qbirthday}\NormalTok{(.}\DecValTok{5}\NormalTok{, }\AttributeTok{coincident =} \DecValTok{3}\NormalTok{)}
\DocumentationTok{\#\# [1] 88}
\end{Highlighting}
\end{Shaded}

\hypertarget{exercuxedcios}{%
\section*{Exercícios}\label{exercuxedcios}}
\addcontentsline{toc}{section}{Exercícios}

\href{https://projects.iq.harvard.edu/files/stat110/files/strategic_practice_and_homework_1.pdf}{Enunciados (pdf).}

\hypertarget{practice}{%
\subsection*{Practice}\label{practice}}
\addcontentsline{toc}{subsection}{Practice}

\hypertarget{norepeat-words}{%
\subsubsection*{4. Norepeat words}\label{norepeat-words}}
\addcontentsline{toc}{subsubsection}{4. Norepeat words}

\begin{rmdbox}
A \emph{norepeatword} is a sequence of at least one (and possibly all) of the usual $26$ letters a, b, c, \ldots,z, with repetitions not allowed.

For example, ``course'' is a \emph{norepeatword}, but ``statistics'' is not.

Order matters, e.g., ``course'' is not the same as ``source''.

A \emph{norepeatword} is chosen randomly, with all \emph{norepeatwords} equally likely. Show that the probability that it uses all $26$ letters is very close to $1/e$.

\end{rmdbox}

\begin{itemize}
\item
  O denominador vai ser o total de todas as \emph{norepeatwords} (NRW), que é a soma de

  \begin{itemize}
  \tightlist
  \item
    NRW de $1$ letra: $26$
  \item
    NRW de $2$ letras: $26 \cdot 25$
  \item
    NRW de $3$ letras: $26 \cdot 25 \cdot 24$
  \item
    $\dots$
  \item
    NRW de $24$ letras: $26 \cdot 25 \cdot 24 \cdot \cdots \cdot 3$
  \item
    NRW de $25$ letras: $26 \cdot 25 \cdot 24 \cdot \cdots \cdot 2$
  \item
    NRW de $26$ letras: $26 \cdot 25 \cdot 24 \cdot \cdots \cdot 1$
  \end{itemize}
\item
  Ou seja,

  \[
  \sum_{k=0}^{25} \frac{26!}{k!}
  \]
\item
  Que é igual a

  \[
  26! 
  \left( 
  1 + \frac{1}{1!} + \frac{1}{2!} + \frac{1}{3!} + \cdots + \frac{1}{25!} 
  \right)
  \]
\item
  O total de NRW que usam as $26$ letras é $26!$.
\item
  A probabilidade procurada é

  \[
  \begin{aligned}
  P 
  &= 
  \frac{26!}{
  26! 
  \left( 
  1 + \frac{1}{1!} + \frac{1}{2!} + \frac{1}{3!} + \cdots + \frac{1}{25!} 
  \right)
  } \\
  &=
  \frac{1}{
  1 + \frac{1}{1!} + \frac{1}{2!} + \frac{1}{3!} + \cdots + \frac{1}{25!} 
  } \\
  &= \frac1e
  \end{aligned}
  \]
\item
  A última igualdade se justifica porque a série de Taylor para $e^x$ é

  \[
  e^x = \sum_{k=0}^\infty \frac{x^k}{k!}
  \]
\item
  Numericamente:

\begin{Shaded}
\begin{Highlighting}[]
\DecValTok{1} \SpecialCharTok{/} \FunctionTok{exp}\NormalTok{(}\DecValTok{1}\NormalTok{)}
\DocumentationTok{\#\# [1] 0,3678794}
\end{Highlighting}
\end{Shaded}

\begin{Shaded}
\begin{Highlighting}[]
\DecValTok{1} \SpecialCharTok{/} \FunctionTok{sum}\NormalTok{(}\DecValTok{1} \SpecialCharTok{/} \FunctionTok{factorial}\NormalTok{(}\DecValTok{0}\SpecialCharTok{:}\DecValTok{25}\NormalTok{))}
\DocumentationTok{\#\# [1] 0,3678794}
\end{Highlighting}
\end{Shaded}
\end{itemize}

\hypertarget{exercuxedcios-do-livro-cap.-1}{%
\subsection*{Exercícios do livro (cap. 1)}\label{exercuxedcios-do-livro-cap.-1}}
\addcontentsline{toc}{subsection}{Exercícios do livro (cap. 1)}

\hypertarget{section}{%
\subsubsection*{13}\label{section}}
\addcontentsline{toc}{subsubsection}{13}

\begin{rmdbox}
A certain casino uses $10$ standard decks of cards mixed together into one big deck, which we will call a superdeck. Thus, the superdeck has $52 \cdot 10 = 520$ cards, with $10$ copies of each card.

How many different $10$-card hands can be dealt from the superdeck? The order of the cards does not matter, nor does it matter which of the original $10$ decks the cards came from. Express your answer as a binomial coefficient.

\end{rmdbox}

\begin{itemize}
\item
  Usando a notação de OLIVEIRA MORGADO et al. (\protect\hyperlink{ref-oliveira-2004-analis}{2004}), onde $\text{CR}_k^n$ é o número de combinações completas de $n$ elementos de $k$ tipos diferentes, a resposta é

  \[
  \text{CR}_{52}^{10} = \binom{52 + 10 - 1}{10} = \binom{61}{10} =
  90.177.170.226
  \]
\item
  Só foi possível usar combinações completas porque a mão tem $10$ cartas, o que faz com que haja, essencialmente, um {\hl{número infinito de cópias de cada um dos $52$ tipos de carta}}. Se a mão tivesse $11$ ou mais cartas, seria impossível que todas as cartas fossem iguais, e este raciocínio não poderia ser usado.
\end{itemize}

\hypertarget{histuxf3rias-e-axiomas}{%
\chapter*{02: Histórias e axiomas}\label{histuxf3rias-e-axiomas}}
\addcontentsline{toc}{chapter}{02: Histórias e axiomas}

\hypertarget{vuxeddeo-1}{%
\section*{Vídeo}\label{vuxeddeo-1}}
\addcontentsline{toc}{section}{Vídeo}

\begin{center} \url{https://youtu.be/FJd_1H3rZGg} \end{center}

\hypertarget{exercuxedcios-1}{%
\section*{Exercícios}\label{exercuxedcios-1}}
\addcontentsline{toc}{section}{Exercícios}

\href{https://projects.iq.harvard.edu/files/stat110/files/strategic_practice_and_homework_1.pdf}{Enunciados (pdf).}

\hypertarget{homework}{%
\subsection*{Homework}\label{homework}}
\addcontentsline{toc}{subsection}{Homework}

\hypertarget{teorema-das-colunas}{%
\subsubsection*{4. Teorema das colunas}\label{teorema-das-colunas}}
\addcontentsline{toc}{subsubsection}{4. Teorema das colunas}

\begin{rmdbox}

\begin{enumerate}
\def\labelenumi{(\alph{enumi})}
\tightlist
\item
  Mostre que
\end{enumerate}

\[
\binom{k}{k} + \binom{k + 1}{k} + \binom{k + 2}{k} + \cdots + 
\binom{n}{k} = \binom{n + 1}{k + 1}
\]

\end{rmdbox}

\begin{itemize}
\item
  O lado direito significa escolher $k + 1$ pessoas dentre $n + 1$ pessoas.
\item
  O truque é {\hl{ordenar as pessoas}} de algum modo.
\item
  Um exemplo concreto, com $n = 4$ e $k = 2$, mostrando que

  \[
  \binom{5}{3} = \binom{4}{2} + \binom{3}{2} + \binom{2}{2}
  \]

  \begin{enumerate}
  \def\labelenumi{\arabic{enumi}.}
  \item
    Vamos chamar as $n + 1$ pessoas de $1, 2, 3, 4, 5$.
  \item
    Grupos de $k + 1$ pessoas onde o menor número é $1$:

    \begin{itemize}
    \tightlist
    \item
      $1, 2, 3$
    \item
      $1, 2, 4$
    \item
      $1, 2, 5$
    \item
      $1, 3, 4$
    \item
      $1, 3, 5$
    \item
      $1, 4, 5$
    \end{itemize}
  \item
    Grupos de $k + 1$ pessoas onde o menor número é $2$:

    \begin{itemize}
    \tightlist
    \item
      $2, 3, 4$
    \item
      $2, 3, 5$
    \item
      $2, 4, 5$
    \end{itemize}
  \item
    Grupos de $k + 1$ pessoas onde o menor número é $3$:

    \begin{itemize}
    \tightlist
    \item
      $3, 4, 5$
    \end{itemize}
  \end{enumerate}
\item
  No caso geral, vamos ordenar as $n + 1$ pessoas, rotulando-as como

  \[
  a_0, a_1, a_2, \dots, a_n
  \]
\item
  Como a ordem {\hl{dentro de cada grupo}} não importa, vamos escolher primeiro um elemento para ser o de {\hl{menor índice do grupo}} e escolher os restantes $k$ elementos dentre os elementos de índice maior do que o primeiro.
\item
  Se escolhermos $a_0$ como o de menor índice, temos $n \choose k$ modos de escolher os restantes.
\item
  Se escolhermos $a_1$ como o de menor índice, temos $n - 1 \choose k$ modos de escolher os restantes.
\item
  $\dots$
\item
  Se escolhermos $a_{n - (k+1)}$ como o de menor índice, temos $k + 1 \choose k$ modos de escolher os restantes.
\item
  Se escolhermos $a_{n - k}$ como o de menor índice, temos $k \choose k$ modos de escolher os restantes.
\end{itemize}

\begin{rmdbox}

\begin{enumerate}
\def\labelenumi{(\alph{enumi})}
\setcounter{enumi}{1}
\tightlist
\item
  Suppose that a large pack of Haribo gummi bears can have anywhere between $30$ and $50$ gummi bears. There are $5$ delicious flavors. How many possibilities are there for the composition of such a pack of gummi bears?
\end{enumerate}

\end{rmdbox}

\begin{itemize}
\item
  Usando a notação de OLIVEIRA MORGADO et al. (\protect\hyperlink{ref-oliveira-2004-analis}{2004}), onde $\text{CR}_k^n$ é o número de combinações completas de $n$ elementos de $k$ tipos diferentes, a resposta é

  \[
  \begin{aligned}
    \text{CR}_5^{30} + \text{CR}_5^{31} + \cdots + \text{CR}_5^{50} 
    &=
    \binom{34}{4} + \binom{35}{4} + \cdots + \binom{54}{4} \\
    &= \binom{55}{5} - 
    \left[ 
    \binom{33}{4} + \binom{32}{4} + \cdots + \binom{4}{4}
    \right] \\
    &= \binom{55}{5} - \binom{34}{5}
  \end{aligned}
  \]
\end{itemize}

\hypertarget{exercuxedcios-do-livro-cap.-1-1}{%
\subsection*{Exercícios do livro (cap. 1)}\label{exercuxedcios-do-livro-cap.-1-1}}
\addcontentsline{toc}{subsection}{Exercícios do livro (cap. 1)}

\hypertarget{section-1}{%
\subsubsection*{17}\label{section-1}}
\addcontentsline{toc}{subsubsection}{17}

\begin{rmdbox}
\[
\sum_{k = 0}^n \binom{n}{k}^2 = \binom{2n}{n}
\]

\end{rmdbox}

\begin{itemize}
\item
  Quero escolher $n$ pessoas dentre $2n$ pessoas (lado direito).
\item
  Divido as $2n$ pessoas em dois grupos de $n$ cada.
\item
  Para $k \in \{0, 1, \ldots n\}$:

  \begin{itemize}
  \item
    Escolho $k$ pessoas do primeiro grupo --- $\binom{n}{k}$ --- para entrar.
  \item
    Escolho $k$ pessoas do segundo grupo --- $\binom{n}{k}$ --- para {\hl{não}} entrar.
  \item
    Para este valor de $k$, tenho $\binom{n}{k}^2$ maneiras de selecionar $n$ pessoas, com $k$ pessoas do primeiro grupo e $n-k$ pessoas do segundo.
  \end{itemize}
\end{itemize}

\hypertarget{section-2}{%
\subsubsection*{18}\label{section-2}}
\addcontentsline{toc}{subsubsection}{18}

\begin{rmdbox}
\[
\sum_{k = 1}^n k\binom{n}{k}^2 = n\binom{2n-1}{n-1}
\]

\end{rmdbox}

Usamos o mesmo raciocínio do exercício anterior, com a seguinte modificação:

\begin{itemize}
\item
  Temos $2n$ pessoas, divididas em dois grupos de $n$, como antes.
\item
  Como antes, quero escolher $n$ pessoas dentre as $2n$.
\item
  Mas agora, para cada escolha, quero designar uma das $n$ pessoas do primeiro grupo como chefe (i.e., sempre haverá pelo menos uma pessoa do primeiro grupo). Isto corresponde ao fator $n$ no lado direito.
\item
  Escolhido o chefe, preciso escolher $n - 1$ pessoas dentre as $2n - 1$ restantes ($n - 1$ do primeiro grupo, $n$ do segundo). Isto corresponde ao segundo fator do lado direito.
\item
  Do lado esquerdo, para $k \in \{1, 2, \ldots, n \}$:

  \begin{itemize}
  \item
    Vou escolher $k$ pessoas do primeiro grupo --- $\binom{n}{k}$ --- para entrar.
  \item
    Dentre elas, vou escolher um chefe --- $k$.
  \item
    Vou escolher $k$ pessoas do segundo grupo --- $\binom{n}{k}$ --- para {\hl{não}} entrar.
  \end{itemize}
\end{itemize}

\hypertarget{section-3}{%
\subsubsection*{19}\label{section-3}}
\addcontentsline{toc}{subsubsection}{19}

\begin{rmdbox}
\[
\sum_{k=2}^n \binom k2 \binom{n-k+2}{2} = \binom{n+3}{5}, \quad \forall n \geq 2
\]

\end{rmdbox}

\hypertarget{section-4}{%
\subsubsection*{22}\label{section-4}}
\addcontentsline{toc}{subsubsection}{22}

\begin{rmdbox}

Para provar

\[
1^3 + 2^3 + \cdots + n^3 = (1 + 2 + \cdots + n)^2
\]

\begin{enumerate}
\def\labelenumi{\alph{enumi}.}
\item
  Vamos provar

  \[
  1 + 2 + \cdots + n = \binom{n+1}{2}
  \]
\item
  E vamos provar

  \[
  1^3 + 2^3 + \cdots + n^3 = 
  6\binom{n+1}{4} + 6\binom{n+1}{3} + \binom{n+1}{2}
  \]
\end{enumerate}

\end{rmdbox}

\hypertarget{problema-do-aniversuxe1rio-propriedades}{%
\chapter*{03: Problema do aniversário, propriedades}\label{problema-do-aniversuxe1rio-propriedades}}
\addcontentsline{toc}{chapter}{03: Problema do aniversário, propriedades}

\hypertarget{vuxeddeo-2}{%
\section*{Vídeo}\label{vuxeddeo-2}}
\addcontentsline{toc}{section}{Vídeo}

\begin{center} \url{https://youtu.be/LZ5Wergp_PA} \end{center}

\hypertarget{exercuxedcios-2}{%
\section*{Exercícios}\label{exercuxedcios-2}}
\addcontentsline{toc}{section}{Exercícios}

\href{???}{Enunciados (pdf).}

\hypertarget{homework-1}{%
\subsection*{Homework}\label{homework-1}}
\addcontentsline{toc}{subsection}{Homework}

\hypertarget{exercuxedcios-do-livro-cap.-1-2}{%
\subsection*{Exercícios do livro (cap. 1)}\label{exercuxedcios-do-livro-cap.-1-2}}
\addcontentsline{toc}{subsection}{Exercícios do livro (cap. 1)}

\hypertarget{section-5}{%
\subsubsection*{26}\label{section-5}}
\addcontentsline{toc}{subsubsection}{26}

\hypertarget{section-6}{%
\subsubsection*{27}\label{section-6}}
\addcontentsline{toc}{subsubsection}{27}

\hypertarget{section-7}{%
\subsubsection*{57}\label{section-7}}
\addcontentsline{toc}{subsubsection}{57}

\hypertarget{section-8}{%
\subsubsection*{61}\label{section-8}}
\addcontentsline{toc}{subsubsection}{61}

\hypertarget{section-9}{%
\subsubsection*{62}\label{section-9}}
\addcontentsline{toc}{subsubsection}{62}

\hypertarget{referuxeancias}{%
\chapter*{Referências}\label{referuxeancias}}
\addcontentsline{toc}{chapter}{Referências}

\hypertarget{refs}{}
\begin{CSLReferences}{0}{1}
\leavevmode\vadjust pre{\hypertarget{ref-devlin-2010-pascal-fermat}{}}%
DEVLIN, K. \href{https://doi.org/10.5951/mt.103.8.0579}{The {P}ascal-{F}ermat Correspondence: How Mathematics Is Really Done}. \textbf{The Mathematics Teacher}, v. 103, n. 8, p. 579--582, abr. 2010.

\leavevmode\vadjust pre{\hypertarget{ref-oliveira-2004-analis}{}}%
OLIVEIRA MORGADO, A. C. DE et al. \textbf{An{á}lise combinat{ó}ria e probabilidade}. Rio de Janeiro: Impa / Vitae, 2004.

\end{CSLReferences}

\end{document}
